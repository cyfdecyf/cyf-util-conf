% $Header: /cvsroot/latex-beamer/latex-beamer/solutions/generic-talks/generic-ornate-15min-45min.en.tex,v 1.4 2004/10/07 20:53:08 tantau Exp $

\documentclass[unicode, hyperref={pdfpagemode=None}]{beamer}

% This file is a solution template for:

% - Giving a talk on some subject.
% - The talk is between 15min and 45min long.
% - Style is ornate.

\mode<presentation>
{
\usetheme{JuanLesPins}
%\usefonttheme[onlylarge]{structuresmallcapsserif}
%\usefonttheme[onlysmall]{structurebold}

% Complete color theme
%\usecolortheme[overlystylish]{albatross}
%\usecolortheme{orchid}

% Outer color theme
%\usecolortheme{whale}
%\usecolortheme{seahorse}

% Inner color theme
%\usecolortheme{rose}
%\usecolortheme{lily}

%\setbeamercolor{frametitle}{fg=black,bg=white}

\setbeamercovered{transparent}
% or whatever (possibly just delete it)

\definecolor{topcolour}{rgb}{0.98,0.98,0.913}
\definecolor{middlecolour}{rgb}{0.952,0.952,0.952}
\definecolor{bottomcolor}{rgb}{0.901,0.901,1}
\setbeamertemplate{background canvas}[vertical shading]
[bottom=bottomcolor, middle=middlecolour, top=topcolour]
}

%\usepackage{CJKutf8}
\usepackage{syntonly}
\usepackage{alltt}
\usepackage{amsmath}
\usepackage{amssymb}
\usepackage{amsmath,amssymb,amsfonts}
\usepackage{times}
\usepackage{multimedia}
\usepackage[T1]{fontenc}
% Or whatever. Note that the encoding and the font should match. If T1
% does not look nice, try deleting the line with the fontenc.

%\syntaxonly

\begin{document}
%\begin{CJK*}{UTF8}{song}
\title
{Title}

\subtitle % (optional, use only with long paper titles)
{}

\author % (optional, use only with lots of authors)
{陈宇飞\\
\quad \\
\scriptsize{Email: cyfdecyf@gmail.com}}

% - Use the \inst{?} command only if the authors have different
%   affiliation.
\institute[计算机科学与技术学院] % (optional, but mostly needed)
{
复旦大学
}

\date {} % (optional)

\subject{Talks}
% This is only inserted into the PDF information catalog. Can be left
% out.

%\pgfdeclareimage[height=.35cm]{logo}{fsa-logo}
%\logo{\pgfuseimage{logo}}

% If you have a file called "university-logo-filename.xxx", where xxx
% is a graphic format that can be processed by latex or pdflatex,
% resp., then you can add a logo as follows:

% \pgfdeclareimage[height=0.5cm]{university-logo}{university-logo-filename}
% \logo{\pgfuseimage{university-logo}}

% Delete this, if you do not want the table of contents to pop up at
% the beginning of each subsection:
\AtBeginSection[] {
\begin{frame}<beamer>
    \frametitle{Outline}
    \tableofcontents[currentsection]
\end{frame}
}

\AtBeginSubsection[] {
\begin{frame}<beamer>
    \frametitle{Outline}
    \tableofcontents[currentsection,currentsubsection]
\end{frame}
}

% If you wish to uncover everything in a step-wise fashion, uncomment
% the following command:

%\beamerdefaultoverlayspecification{<+->}

\begin{frame}
    \titlepage
\end{frame}

% Since this a solution template for a generic talk, very little can
% be said about how it should be structured. However, the talk length
% of between 15min and 45min and the theme suggest that you stick to
% the following rules:

% - Exactly two or three sections (other than the summary).
% - At *most* three subsections per section.
% - Talk about 30s to 2min per frame. So there should be between about
%   15 and 30 frames, all told.

\begin{frame}
    \frametitle{What I will talk about}
    \begin{itemize}
        \item
            \pause
    \end{itemize}
\end{frame}

\begin{frame}
    \frametitle{References and online resources}
    \begin{thebibliography}{9}
        \bibitem{RFCutf8} IETF RFC2279, \textit{UTF-8, a transformation format of
            ISO 10646}
    \end{thebibliography}
\end{frame}

\begin{frame}
    \begin{center}
        {\huge 
        %\nbs \\
        Thanks for your time }
    \end{center}
\end{frame}

\newpage
%\end{CJK*}
\end{document}

